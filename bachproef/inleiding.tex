%%=============================================================================
%% Inleiding
%%=============================================================================

\chapter{\IfLanguageName{dutch}{Inleiding}{Introduction}}%
\label{ch:inleiding}

% De inleiding moet de lezer net genoeg informatie verschaffen om het onderwerp te begrijpen en in te zien waarom de onderzoeksvraag de moeite waard is om te onderzoeken. In de inleiding ga je literatuurverwijzingen beperken, zodat de tekst vlot leesbaar blijft. Je kan de inleiding verder onderverdelen in secties als dit de tekst verduidelijkt. Zaken die aan bod kunnen komen in de inleiding~\autocite{Pollefliet2011}:

% \begin{itemize}
%   \item context, achtergrond
%   \item afbakenen van het onderwerp
%   \item verantwoording van het onderwerp, methodologie
%   \item probleemstelling
%   \item onderzoeksdoelstelling
%   \item onderzoeksvraag
%   \item \ldots
% \end{itemize}

In een tijdperk waarin digitale transformatie en hybride cloudomgevingen de norm zijn geworden, is een geïntegreerde aanpak van netwerk en beveiliging centraal komen te staan voor softwarebedrijven. 
Traditionele netwerkmodellen, gebaseerd op perimeterbescherming via firewalls en VPNs, volstaan niet langer in een wereld van directe cloud-toegang, telewerken en verspreide applicaties. 

\vspace{2ex}

SASE (Secure Access Service Edge) biedt een geïntegreerd framework dat netwerk- en beveiligingsfuncties combineert in één cloudgebaseerd servicemodel. Voor softwarebedrijven met gevoelige klantdata, is deze aanpak essentieel om compliance te waarborgen en vertrouwen te behouden.

\vspace{2ex}

In dit onderzoek wordt de implementatie van een SASE (Secure Access Service Edge) oplossing, met de focus op Netskope, onderzocht.
\section{\IfLanguageName{dutch}{Probleemstelling}{Problem Statement}}%
\label{sec:probleemstelling}

% Uit je probleemstelling moet duidelijk zijn dat je onderzoek een meerwaarde heeft voor een concrete doelgroep. De doelgroep moet goed gedefinieerd en afgelijnd zijn. Doelgroepen als ``bedrijven,'' ``KMO's'', systeembeheerders, enz.~zijn nog te vaag. Als je een lijstje kan maken van de personen/organisaties die een meerwaarde zullen vinden in deze bachelorproef (dit is eigenlijk je steekproefkader), dan is dat een indicatie dat de doelgroep goed gedefinieerd is. Dit kan een enkel bedrijf zijn of zelfs één persoon (je co-promotor/opdrachtgever).

Het onderzochte softwarebedrijf is gespecialiseerd in digitale producten. Het kampt met toenemende uitdagingen rond netwerktoegang en security in zijn hybride cloudinfrastructuur.

\vspace{2ex}

\textbf{Specifieke pijnpunten zijn:}

\begin{itemize}
  \item User \& Device: Authenticatie en autorisatie van gebruikers en apparaten.
	\item Application \& Workload: Applicatie-specifieke toegangscontrole en monitoring.
  \item Data: Databescherming en classificatie
  \item Connectivity: veilige verbinding tussen cloud- en at home-omgevingen.
\end{itemize}

Om deze uitdagingen aan te pakken, heeft het bedrijf gekozen voor het Netskopeplatform om een SASE-architectuur te implementeren. Deze keuze vereist echter een gedetailleerd inzicht in de technische mogelijkheden van Netskope en een op maat gemaakte implementatiestrategie die zowel netwerk- als beveiligingsaspecten integreert.

\section{\IfLanguageName{dutch}{Onderzoeksvraag}{Research question}}%
\label{sec:onderzoeksvraag}

% Wees zo concreet mogelijk bij het formuleren van je onderzoeksvraag. Een onderzoeksvraag is trouwens iets waar nog niemand op dit moment een antwoord heeft (voor zover je kan nagaan). Het opzoeken van bestaande informatie (bv. ``welke tools bestaan er voor deze toepassing?'') is dus geen onderzoeksvraag. Je kan de onderzoeksvraag verder specifiëren in deelvragen. Bv.~als je onderzoek gaat over performantiemetingen, dan 

Hoe kan een SASE-architectuur worden geïmplementeerd binnen de hybride cloudomgeving van een softwarebedrijf gebruikmakend van het Netskope platform?

\vspace{2ex}

Deze vraag wordt opgesplitst in probleem- en oplossingsdomein:

\textbf{Probleemdomein}
\begin{itemize}
  \item Welke netwerk- en security-uitdagingen in de huidige infrastructuur kunnen worden aangepakt via Netskope's SASE-functionaliteiten?
  \item Welke tekortkomingen in de huidige perimeterbeveiliging vereisen een Zero Trust-aanpak?
\end{itemize}

\vspace{2ex}

\textbf{Oplossingsdomein}
\begin{itemize}
  \item Welke Netskope-componenten zijn nodig voor een succesvolle implementatie?
  \item Hoe kan Netskope geoptimaliseerd worden voor een hybride omgeving, inclusief integratie met bestaande systemen?
\end{itemize}

\vspace{2ex}

\textbf{Deelvragen}
\begin{itemize}
  \item Wat is de huidige situatie van het bedrijf op vlak van security en welke risico’s zijn er?
  \item Welke Netskope-functionaliteiten zijn relevant voor de implementatie van Zero Trust?
  \item Hoe kan Netskope geïntegreerd worden met bestaande systemen en processen?
  \item Hoe kan de implementatie van Netskope gevalideerd worden?
\end{itemize}

\section{\IfLanguageName{dutch}{Onderzoeksdoelstelling}{Research objective}}%
\label{sec:onderzoeksdoelstelling}

% Wat is het beoogde resultaat van je bachelorproef? Wat zijn de criteria voor succes? Beschrijf die zo concreet mogelijk. Gaat het bv.\ om een proof-of-concept, een prototype, een verslag met aanbevelingen, een vergelijkende studie, enz.

Het doel van deze proef is het ontwikkelen van een proof of concept dat de haalbaarheid en voordelen van de SASE-architectuur aantoont. Het resultaat gaat verder dan een geschreven scriptie en omvat dus ook een implementatie die de werking van het systeem aantoont.

\vspace{2ex}

Deze PoC zal worden geëvalueerd door de interne IT afdeling in de laatste 2 weken van dit onderzoek.

\section{\IfLanguageName{dutch}{Opzet van deze bachelorproef}{Structure of this bachelor thesis}}%
\label{sec:opzet-bachelorproef}

% Het is gebruikelijk aan het einde van de inleiding een overzicht te
% geven van de opbouw van de rest van de tekst. Deze sectie bevat al een aanzet
% die je kan aanvullen/aanpassen in functie van je eigen tekst.

De rest van deze bachelorproef is als volgt opgebouwd:

\vspace{2ex}

In Hoofdstuk~\ref{ch:literatuurstudie} wordt een overzicht gegeven van de stand van zaken binnen het onderzoeksdomein, op basis van een literatuurstudie.

\vspace{2ex}

In Hoofdstuk~\ref{ch:methodologie} wordt de methodologie toegelicht en worden de gebruikte onderzoekstechnieken besproken om een antwoord te kunnen formuleren op de onderzoeksvragen.

\vspace{2ex}

In Hoofdstuk~\ref{ch:requirements analyse} worden de functionele en niet-functionele vereisten voor de SASE-architectuur in de hybride cloudomgeving van het softwarebedrijf geanalyseerd. Dit hoofdstuk legt de basis voor de implementatie door de specifieke behoeften en beperkingen van het bedrijf te identificeren.

\vspace{2ex}

In Hoofdstuk~\ref{ch:proof of concept preparation} wordt de voorbereiding van de proof of concept besproken, inclusief de technische vereisten en de opzet van de implementatie.

\vspace{2ex}

In Hoofdstuk~\ref{ch:proof of concept implementation} wordt de implementatie van de proof of concept besproken, met aandacht voor de uitvoering en de behaalde resultaten.

\vspace{2ex}

In Hoofdstuk~\ref{ch:resultaten} worden de resultaten van de proof of concept gepresenteerd en geanalyseerd, met een focus op de effectiviteit van de SASE-architectuur in de hybride cloudomgeving.

\vspace{2ex}

In Hoofdstuk~\ref{ch:conclusie}, tenslotte, wordt de conclusie gegeven en een antwoord geformuleerd op de onderzoeksvragen. Daarbij wordt ook een aanzet gegeven voor toekomstig onderzoek binnen dit domein.