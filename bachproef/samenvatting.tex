%%=============================================================================
%% Samenvatting
%%=============================================================================

% TODO: De "abstract" of samenvatting is een kernachtige (~ 1 blz. voor een
% thesis) synthese van het document.
%
% Een goede abstract biedt een kernachtig antwoord op volgende vragen:
%
% 1. Waarover gaat de bachelorproef?
% 2. Waarom heb je er over geschreven?
% 3. Hoe heb je het onderzoek uitgevoerd?
% 4. Wat waren de resultaten? Wat blijkt uit je onderzoek?
% 5. Wat betekenen je resultaten? Wat is de relevantie voor het werkveld?
%
% Daarom bestaat een abstract uit volgende componenten:
%
% - inleiding + kaderen thema
% - probleemstelling
% - (centrale) onderzoeksvraag
% - onderzoeksdoelstelling
% - methodologie
% - resultaten (beperk tot de belangrijkste, relevant voor de onderzoeksvraag)
% - conclusies, aanbevelingen, beperkingen
%
% LET OP! Een samenvatting is GEEN voorwoord!

%%---------- Nederlandse samenvatting -----------------------------------------
%
% TODO: Als je je bachelorproef in het Engels schrijft, moet je eerst een
% Nederlandse samenvatting invoegen. Haal daarvoor onderstaande code uit
% commentaar.
% Wie zijn bachelorproef in het Nederlands schrijft, kan dit negeren, de inhoud
% wordt niet in het document ingevoegd.

\IfLanguageName{english}{%
\selectlanguage{dutch}
\chapter*{Samenvatting}
\selectlanguage{english}
}{}

%%---------- Samenvatting -----------------------------------------------------
% De samenvatting in de hoofdtaal van het document

\chapter*{\IfLanguageName{dutch}{Samenvatting}{Abstract}}

Deze bachelorproef behandelt de implementatie van een Secure Access Service Edge (SASE) architectuur binnen de hybride cloudomgeving van een software agency bedrijf, gebruikmakend van het Netskope platform. Het onderzochte softwarebedrijf kampt met toenemende uitdagingen op het gebied van netwerktoegang en beveiliging in zijn hybride cloud infrastructuur. De huidige perimetergebaseerde beveiliging biedt onvoldoende flexibiliteit en bescherming voor de moderne verspreide werkomgeving.

\vspace{2ex}

De centrale onderzoeksvraag luidt: "Hoe kan een SASE-architectuur worden geïmplementeerd binnen de hybride cloudomgeving van een softwarebedrijf gebruikmakend van het Netskope platform?". De doelstelling is het ontwikkelen van een functioneel proof of concept dat de haalbaarheid en voordelen van SASE-architectuur aantoont voor hybride cloudomgevingen.

\vspace{2ex}

De methodologie bestaat uit vier systematische fasen: literatuuronderzoek, requirements analyse, proof of concept ontwikkeling en validatie. Het ontwikkelde proof of concept implementeert een volledige Netskope SASE-architectuur met vijf kerncomponenten: Netskope Client, Next Gen Secure Web Gateway, Cloud Access Security Broker, Zero Trust Network Access en Netskope Publisher. Geïntegreerd met bestaande systemen zoals Okta en Jamf.

\vspace{2ex}

De resultaten tonen aan dat de SASE-architectuur een effectieve oplossing biedt voor moderne beveiligingsuitdagingen. De proof of concept demonstreert verbeterde beveiliging door geïntegreerde Zero Trust implementatie, verhoogde operationele efficiëntie door geautomatiseerde policy management en verbeterde gebruikerservaring zonder prestatieverlies. De implementatie biedt het software agency bedrijf een schaalbare security architectuur die aansluit bij groeiambities en moderne werkpraktijken.
