%%=============================================================================
%% Voorwoord
%%=============================================================================

\chapter*{\IfLanguageName{dutch}{Woord vooraf}{Preface}}%
\label{ch:voorwoord}

%% TODO:
%% Het voorwoord is het enige deel van de bachelorproef waar je vanuit je
%% eigen standpunt (``ik-vorm'') mag schrijven. Je kan hier bv. motiveren
%% waarom jij het onderwerp wil bespreken.
%% Vergeet ook niet te bedanken wie je geholpen/gesteund/... heeft

Toen ik de mogelijkheid kreeg om onderzoek te doen naar de implementatie van een SASE-architectuur binnen een reële bedrijfsomgeving, was ik direct enthousiast. Zowel over de praktische relevantie als de technische uitdaging die dit met zich zou brengen.

\vspace{2ex}

De keuze voor dit onderwerp werd mede geïnspireerd door de vaststelling dat traditionele perimetergebaseerde beveiligingsmodellen steeds minder effectief worden. Als toekomstig IT-professional wil ik begrijpen hoe moderne beveiligingsarchitecturen zoals Zero Trust praktisch kunnen worden geïmplementeerd om organisaties te helpen bij hun digitale transformatie.

\vspace{2ex}

Gedurende dit onderzoek heb ik kunnen ervaren hoe complex en veelomvattend de implementatie van een SASE-architectuur werkelijk is. Het ontwikkelen van een functioneel proof of concept met het Netskope platform heeft mij niet alleen technische inzichten gegeven, maar ook het belang van stakeholder management en change management binnen IT-projecten doen inzien. Ik was enorm onder de indruk over de uitdaging om de werknemers achter de nieuwe technologie te krijgen en hen te overtuigen van de voordelen van deze implementatie.

\vspace{2ex}

Ik wil graag een aantal personen bedanken die essentieel zijn geweest voor het tot stand komen van deze bachelorproef. Allereerst mijn promotor dr. Pieter-Jan Maenhaut, voor de waardevolle begeleiding, kritische feedback en het bewaken van de academische kwaliteit van dit werk. Mijn co-promotor Louis De Jaeger verdient bijzondere dank voor de inhoudelijke expertise en de mogelijkheid om dit onderzoek uit te voeren binnen een authentieke bedrijfscontext.

\vspace{2ex}

Verder ben ik dankbaar aan de leden van Team IT die tijd hebben vrijgemaakt voor interviews en feedback. Hun praktische inzichten waren onmisbaar voor het ontwikkelen van een realistische implementatiestrategie. Ook de vertegenwoordigers van Netskope wil ik bedanken voor hun technische ondersteuning en documentatie die cruciaal waren voor het succes van het proof of concept.

\vspace{2ex}

Deze bachelorproef markeert het einde van mijn opleiding toegepaste informatica aan HOGENT, maar tegelijkertijd het begin van een professionele carrière. Een waar ik zal blijven streven naar innovatieve oplossingen binnen de IT-sector.
