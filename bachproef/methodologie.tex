%%=============================================================================
%% Methodologie
%%=============================================================================

\chapter{\IfLanguageName{dutch}{Methodologie}{Methodology}}%
\label{ch:methodologie}

%% TODO: In dit hoofstuk geef je een korte toelichting over hoe je te werk bent
%% gegaan. Verdeel je onderzoek in grote fasen, en licht in elke fase toe wat
%% de doelstelling was, welke deliverables daar uit gekomen zijn, en welke
%% onderzoeksmethoden je daarbij toegepast hebt. Verantwoord waarom je
%% op deze manier te werk gegaan bent.
%% 
%% Voorbeelden van zulke fasen zijn: literatuurstudie, opstellen van een
%% requirements-analyse, opstellen long-list (bij vergelijkende studie),
%% selectie van geschikte tools (bij vergelijkende studie, "short-list"),
%% opzetten testopstelling/PoC, uitvoeren testen en verzamelen
%% van resultaten, analyse van resultaten, ...
%%
%% !!!!! LET OP !!!!!
%%
%% Het is uitdrukkelijk NIET de bedoeling dat je het grootste deel van de corpus
%% van je bachelorproef in dit hoofstuk verwerkt! Dit hoofdstuk is eerder een
%% kort overzicht van je plan van aanpak.
%%
%% Maak voor elke fase (behalve het literatuuronderzoek) een NIEUW HOOFDSTUK aan
%% en geef het een gepaste titel.

Dit onderzoek volgt een systematische aanpak die bestaat uit drie hoofdfasen: literatuuronderzoek, technische analyse, en proof of concept ontwikkeling. De totale onderzoeksduur bedraagt 14 weken.

\subsection{Fase 1: Literatuuronderzoek (3 weken)}
Fase 1 van het project betreft een literatuuronderzoek van drie weken waarin de eerste twee weken worden besteed aan het analyseren van Netskope’s technische documentatie en architectuur, het bestuderen van Netskope’s Zero Trust Reference Architecture en het onderzoeken van best practices voor Netskope implementaties. 
In de derde week worden de bevindingen gesynthetiseerd en wordt een Netskope-specifiek implementatieplan opgesteld, wat resulteert in het eerste deliverable: een implementatieplan met Netskope configuratierichtlijnen.

\subsection{Fase 2: Requirements analyse en interviews (4 weken)}
Interviews met IT-team om de huidige situatie te analyseren, dit zal een beter inzicht geven in de huidige security risico's en de vereisten voor een Zero Trust implementatie. Deliverable: een oplijsting van de requirements en een
verslag van de interviews.

\subsection{Fase 3: Proof of concept ontwikkeling (5 weken)}
De volgende zeven weken zijn voorzien voor de daadwerkelijke ontwikkeling van het
proof of concept. Hierbij wordt gebruik gemaakt van geschikte tools en technolo-
gieën die zijn gevonden tijdens de literatuurstudie en ontwerpfase. De PoC zal aan de onderzochte requirements voldoen en zal het onderzochte probleem oplossen. De deliverable is een functioneel proof of concept van het netskope systeem.

\subsection{Fase 4: Validatie en evaluatie (2 weken)}
De ontwikkelde proof of concept wordt geëvalueerd en gevalideerd. Deze fase omvat testen van de functionaliteit, prestaties en
gebruikers feedback. Eventuele aanpassingen worden doorgevoerd. De einddeli-
verable omvat een afgewerkt proof of concept.

Deze methodologie is specifiek afgestemd op de implementatie van Zero Trust via het Netskope platform, waarbij elke fase concrete technische deliverables oplevert. De focus ligt op het correct configureren en valideren van Netskope's security features, met voldoende tijd voor iteratie en optimalisatie van de implementatie.
