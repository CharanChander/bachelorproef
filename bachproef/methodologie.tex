%%=============================================================================
%% Methodologie
%%=============================================================================

\chapter{\IfLanguageName{dutch}{Methodologie}{Methodology}}%
\label{ch:methodologie}

%% TODO: In dit hoofstuk geef je een korte toelichting over hoe je te werk bent
%% gegaan. Verdeel je onderzoek in grote fasen, en licht in elke fase toe wat
%% de doelstelling was, welke deliverables daar uit gekomen zijn, en welke
%% onderzoeksmethoden je daarbij toegepast hebt. Verantwoord waarom je
%% op deze manier te werk gegaan bent.
%% 
%% Voorbeelden van zulke fasen zijn: literatuurstudie, opstellen van een
%% requirements-analyse, opstellen long-list (bij vergelijkende studie),
%% selectie van geschikte tools (bij vergelijkende studie, "short-list"),
%% opzetten testopstelling/PoC, uitvoeren testen en verzamelen
%% van resultaten, analyse van resultaten, ...
%%
%% !!!!! LET OP !!!!!
%%
%% Het is uitdrukkelijk NIET de bedoeling dat je het grootste deel van de corpus
%% van je bachelorproef in dit hoofstuk verwerkt! Dit hoofdstuk is eerder een
%% kort overzicht van je plan van aanpak.
%%
%% Maak voor elke fase (behalve het literatuuronderzoek) een NIEUW HOOFDSTUK aan
%% en geef het een gepaste titel.

Dit onderzoek volgt een systematische aanpak die bestaat uit drie hoofdfasen: literatuuronderzoek, technische analyse, en proof of concept ontwikkeling. De totale onderzoeksduur bedraagt 14 weken.

\section{Fase 1: Literatuuronderzoek (4 weken)}
Fase 1 van het project betreft een literatuuronderzoek van drie weken waarin de technische documentatie van de Netskope architectuur wordt bestudeerd. 

\vspace{2ex}

Ook externe bronnen worden bestudeerd om een beter beeld te krijgen van de SASE architectuur en de verschillende implementaties.

\section{Fase 2: Requirements analyse (1 week)}
Interviews met IT-team om de huidige situatie te analyseren, dit zal een beter inzicht geven in de huidige security risico's en de vereisten voor een Zero Trust implementatie. 
Via deze gesprekken willen we eerst begrijpen hoe het huidige netwerk is opgebouwd, inclusief hoe de verbindingen tussen kantooromgeving en cloudplatformen verlopen. We bekijken welke beveiligingsmaatregelen al aanwezig zijn en hoe goed deze werken in een hybride cloudopstelling. Ook brengen we in kaart welke beveiligingsrisico’s er zijn, vooral op het gebied van toegangsbeheer, gebruikersverificatie en gegevensbescherming.

\vspace{2ex}

Naast de interviews bekijken we ook bestaande documentatie zoals netwerktekeningen, beveiligingsbeleid en rapporten van vroegere incidenten. Door deze combinatie krijgen we een completer beeld. 

\vspace{2ex}

We letten vooral op risico’s die ontstaan door de traditionele “perimeter-gebaseerde” beveiliging, en hoe een Zero Trust aanpak binnen SASE hier verbetering in kan brengen. 

\vspace{2ex}

Deliverable: een oplijsting van de requirements en een overzicht van de huidige situatie.

\section{Fase 3: Proof of concept ontwikkeling (7 weken)}
De volgende zeven weken zijn voorzien voor de daadwerkelijke ontwikkeling van het
proof of concept. Hierbij wordt gebruik gemaakt van geschikte tools en technologieën die zijn gevonden tijdens de literatuurstudie en ontwerpfase.

\vspace{2ex}

De PoC zal aan de onderzochte requirements voldoen en zal het onderzochte probleem oplossen. 

\vspace{2ex}

De deliverable is een functioneel proof of concept van het netskope systeem.

\section{Fase 4: Validatie en evaluatie (2 weken)}
De ontwikkelde proof of concept wordt geëvalueerd en gevalideerd. Deze fase omvat testen van de functionaliteit, prestaties en
gebruikers feedback. Eventuele aanpassingen worden doorgevoerd. 

\vspace{2ex}

De einddeliverable omvat een afgewerkt proof of concept.

\vspace{2ex}

Deze methodologie is specifiek afgestemd op de implementatie van een SASE architectuur via het Netskope platform, waarbij elke fase concrete technische deliverables oplevert. De focus ligt op het correct configureren en valideren van Netskope's security features, met voldoende tijd voor iteratie en optimalisatie van de implementatie.

